The first submission and the final defense of this thesis were separated
by a time span of about three months, during which I applied some
significant changes to the prototype. These changes are mostly feature
enhancements, but I also reflected on my initial position on the ethics
of \emph{tracking}. This postscriptum pays account to both.

As Atom is still in very active development, some changes to the
underlying architecture and the APIs are to be expected. The updated
version of the prototype makes use of a new editor component, which
offers better performance, and a new API for setting so-called
\emph{markers}. Markers depict positions in the text file (row and
column), which are then used by the Scope Inspector to highlight the
current scope. However, the new marker API does not allow to place
hoisting indicators as easily as before, which is why they have been
disabled for the time being.

In response to the feedback gathered during user testing, three more
features were added.

\begin{itemize}
\itemsep1pt\parskip0pt\parsep0pt
\item
  Users can now independently enable and disable all three UI features
  of the Scope Inspector: scope highlighting, the sidebar, and the
  bottom bar. This aligns with the \emph{modularity} characteristic, as
  identified in chapter \ref{exploration}.
\item
  The sidebar can be used to navigate around the source code. By
  clicking on a scope title or a scoped variable, the cursor jumps to
  beginning of the scope or the variable definition, respectively. This
  feature is commonly known as “jump to definition” in IDEs.
\item
  A most critical feature, the \emph{on-the-fly} re-evaluation of the
  scope structure was implemented as well. Once the user has stopped
  typing for \texttt{n} milliseconds (200 by default), the scope
  structure is evaluated again. Formerly, the Scope Inspector would only
  re-evaluate when saving the file.
\end{itemize}

In addition to the modifications described above, some small usability
enhancements were implemented. The bottom bar was made thinner in order
to save vertical screen space, and the individual “breadcrumbs”, which
were formerly buttons, now resemble breadcrumbs on websites more
closely. In the sidebar, a subtle hint is added in case a scope is
empty; formerly, the Scope Inspector would just display the scope’s
title.

The prototype’s new version also reflects a change in my position
towards tracking (in this particular case). The original prototype
offered metrics tracking via opt-in; the new version, however, enables
tracking by default (for new installations) and offers opt-out. I
revised my concerns towards tracking, as all data being tracked are
completely anonymous and can never lead back to a specific user. Quite
the opposite, these data are only useful as aggregates. This way of
collecting metrics is common practice in design research. The privacy of
any users are not invaded using this technique, and its use is clearly
communicated in the repository and the Atom plugin directory.

After this change in policy, the analytics show that the Scope Inspector
is in fact used daily by a number people (as of December 2014, ranging
from about 10 users on the weekend to up to 40 users during a weekday).
Enabling the tracking proves to be invaluable for evaluating usability
from a quantitative perspective. The sidebar is enabled and disabled
more often than any other component in the Scope Inspector; one can
conclude that it is opened \emph{on demand}, whenever a user wants to
have a deeper insight into the scope structure of a JavaScript file.
Users do not use the “jump to definition” functionality much; its
existence should probably be communicated more clearly. The bottom bar,
however, seems to be activated most of the time. A possible reason is
that it gives the user a quick overview about her position in the scope
chain and allows for quick preview of, and navigation to, outer scopes,
without taking up too much screen real estate.
