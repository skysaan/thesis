\chapter{Research Framework}\label{research}

This chapter introduces the research done prior to the design. It
presents a short history of \glspl{ide} and list typical \gls{ui} design
patterns in \glspl{ide}.

\section{History and Purpose of Integrated Development
Environments}\label{history-and-purpose-of-integrated-development-environments}

\begin{quote}
„A programming environment is a user interface for understanding a
program.“ — Bret Victor \citeyear{victor}
\end{quote}

Software development environments have been predecessed by general text
editors, starting with several projects at the Xerox \gls{parc}. Douglas
Engelbart created the text editor for the NLS system (oNLine System)
which allowed editing with direct manipulation and \gls{wysiwyg}. In the
\emph{Gypsy} text editor, Larry Tesler first integrated modeless moving
of text, which is known as \emph{Copy \& Paste} \cite{moggridge}. Text
editors with those functionalities are now the core of any software
development environment.

Later, while working with Alan Kay, Tesler created the first class
browser for the Smalltalk programming language. Class browsers are used
to look at programs not as of textual source code, but as of logical
entities of a programming language (for example classes and methods).
The Smalltalk class browser was therefore the first software
specifically written for creating software, and a predecessor to any
modern development environment.

\emph{\glspl{ide}} integrate text editors (due to their specific purpose
also referred to here as \emph{code editors}) with other software
development tools. Typically, those tools include compilers, build
systems, syntax highlighters, autocompletion, debuggers, and symbol
browsers. The first \ac{ide} is said to be \emph{Maestro I} by Softlab,
a whole terminal dedicated to integrating various development tasks
\cite{maestro}.

\section{IDEs compared to Text
Editors}\label{ides-compared-to-text-editors}

It is difficult to delimit the term „\acl{ide}“ and contrast it with
text editor that are mainly used for programming. \citename{reynolds}
formulates a basic definition:

\begin{quote}
„What the different is between a text editor and an IDE – to me at least
– is that an IDE understands the language, whereas the text editor
understands text.“ \citeyear{reynolds}
\end{quote}

In his article, \citename{reynolds} tries to make a point against the
use of text editors for programming by stating that an IDE brings
„forward an understanding of the underlying language and the structure
of code, and puts it front-and-centre in your working environment.“
\citeyear{reynolds} While certainly being correct with this point, he
ignores situations where the „understanding of the underlying language
and the structure of code“ is either not
wanted\footnote{For example, because it may collide with other features that have a higher priority for the respective developer.}
or not possible to achieve.

According to \citeasnoun{lynch} the latter is often the case in web
front-end development. Through working with lots of different file types
and programming languages, neither of which dictates a certain structure
(in opposition to many static languages like Java), an IDE can only have
a limited understanding about the structure of the code.
\citename{lynch} also states that IDEs „tend to be built with a workflow
in mind“, therefore being seen as opinionated.

In other words, IDEs and text editors seem to follow different,
contradirectional approaches. While the latter is built around a central
paradigm (text editing) and usually comes with a minimal program core
that is extendable to personal likes, IDEs tend to offer everything ‚out
of the box‘ as a one-stop solution.

For this thesis, the distinction only plays a subordinate role, as most
of the concepts and ideas discussed here can be applied to both kinds of
software. However, it is important to clarify that both are adressed
when using, interchangably, any of the following terms: \emph{Integrated
Development Environment (IDE)}, \emph{development environment},
\emph{software development environment}, \emph{programming environment}.

\section{The Current Landscape of Development
Environments}\label{the-current-landscape-of-development-environments}

The IDE landscape is today more differentiated than ever, ranging from
minimal, purpose-specific editors to full-fledged, general-purpose,
commercial development environments. This diversity can also be seen in
the survey results (see chapter \ref{exploration}).

On the commercial side, Microsoft Visual Studio is the monopolistic
development environment for the .Net platform. The Java platform is
dominated mostly by open source IDEs, such as Eclipse and NetBeans,
although they and their derivatives are widely used for other
programming languages as well. More specialized are the Processing and
Arduino environments. JavaScript and web frontend developers, however,
often use more minimalistic code editors, for example Vim or Sublime
Text.

Those different IDEs serve the needs of different developers and
development situations. But still, it seems like there are many niches
that are yet to be filled with new environments. Especially the area of
web development (frontend development) experiences many new products
quite often, which is possibly related to JavaScript’s growing
importance as the language of the web. The latest additions to the row
of web-focused IDEs include Github’s Atom, Adobe’s Brackets and Eclipse
Orion, all of which are based on Node.js and other web technologies.

There are also recent developments in different development paradigms.
\citeasnoun*{deline} and \citeasnoun*{bragdon} introduce novel user
interface metaphors to structure and navigate program code. \emph{Code
Bubbles} by \citename{bragdon} provides a prototypical implementation,
and a similar concept is adapted by \citeasnoun{granger}. \emph{Code
Thumbnails}, as presented by \citename{deline}, is implemented in
Sublime Text.

\section{Interface and Interaction Patterns in
IDEs}\label{interface-and-interaction-patterns-in-ides}

Many \acl{ui} patterns found in \glspl{ide} are general, well-known
\ac{ui} patterns adapted to a specific purpose. This section gives an
overview on interaction patterns in IDEs that are relevant to this
thesis.

\subsection{User Interface Patterns}\label{user-interface-patterns}

\begin{description}
\item[Code Editor]
Central to every \gls{ide}, a code editor is a specialized text editor,
used for reading and writing program code. It typically features a
\emph{gutter} (see below) and \gls{syntaxhighlighting}. In opposition to
the text editor of a word processor, code editors are not rich text
editors. They also display a monospaced font, which allows to see the
editor content as a grid of rows and columns. With evenly-spaced
columns, due to the monospaced font, code formatting and line
indentation\footnote{In many programming languages, line indentation is an important concept, either as a core syntactical concept or for the sake of readability.}
is made consistent.
\item[Gutter]
The gutter is part of the code editor and describes the narrow space
next to the actual code (usually to the left). Gutters are mainly used
to display line numbers (important for navigation and debugging), but
some provide more advanced features, for example setting
breakpoints\footnote{A feature of the debugger; when set, the program stops at the specified line to allow step-by-step investigation.},
indicating errors in the code through symbols, showing version control
information, or allowing to fold code away in order to either focus or
get an overview.
\item[Panel (sidebar)]
A panel is a rectangular \ac{ui} area used to group together interface
element of similar functionality or other commonalities together. Often,
panels are used on the edges of application windows; if they are on the
left or right side, they may be called \emph{sidebar}. Panels that host
a great number of program functionalities are often called
\emph{toolbar}. Some applications implement \emph{dockable} panels,
which can be moved around and snapped to different areas on the screen.
Another common characteristic is that panels can be resized and
\emph{toggled}, i.e. shown and hidden, on demand.
\item[Status bar]
The status bar is known from many programs, for example web browsers and
word processors. It is a small bar (about one text line of height) at
the bottom of the program window, usually spanning the whole window
width. It is mainly used to display status information and quickly
switch between different application modes (for example „insert“ and
„overwrite“ in word processors).
\end{description}

\subsection{Interactional patterns}\label{interactional-patterns}

\begin{description}
\item[Navigation]
Usually, code can be both browsed and searched for from different
perspectives.

For browsing, most IDEs have a built-in file browser. IDEs that have the
respective understanding of code structure can also offer a more
\emph{logical} way of navigating, for examply by symbolic entities like
modules, classes and methods. Those are usually listed in a symbol
browser or class browser. In the Eclipse IDE, the file browser and
symbol browser are combined into one component, called the \emph{project
explorer}.

IDE facilities for searching work analoguously. Files within a project
can be searched for by their name or their content. If the IDE knows
about the symbols of a programming language, those can usually be
searched for as well. Additionally, some IDEs like Eclipse allow the
user to right click on a method call and jump to its definition source
file, if available.
\item[Modes]
In most IDEs, \ac{ui} elements can be shown or hidden, sometimes even
positioned anywhere on the screen. The Eclipse IDE even allows the
creation of completely different \ac{ui} configurations, so-called
\emph{perspectives}. Usually, perspectives are build for a certain task,
e.g. developing or debugging. Text editors like Sublime Text and
Atom\footnote{In Atom, this has to be installed through a package: \url{https://atom.io/packages/zen}}
support a so-called \emph{distraction-free mode}, in which all \acl{ui}
elements are hidden except the editor itself.
\item[Input]
Most modern IDEs are mouse-driven, which means that every goal—except
writing code—can be achieved using the mouse alone. However, as users
proceed to become more familiar and proficient with the IDE, they tend
to utilize keyboard shortcuts to be more efficient. Many IDEs allow the
user to configure keyboard shortcuts freely; others even offer a single
shortcut to reach any menu item through fuzzy searching, for example
Sublime Text.
\item[Execution, Evaluation and Debugging]
Most IDEs allow the user not only to edit a program, but to compile,
run, and debug it from inside the IDE. This has the advantage that any
information related to compilation-time and run-time can be used and
presented in the IDE itself. In its simplest form, compilation errors or
the console output of a program is shown in an extra output area on the
screen. More advanced implementations show debugging or compilation
informatin \emph{inline}. To give an example: if the Java compiler in
Eclipse encounters an error, it lists the error in an extra panel, but
also underlines the affected code with a red line. For more information
concerning the different lifecycle phases of a program, please refer to
chapter \fullref{concepts}.
\end{description}
