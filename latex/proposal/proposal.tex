\documentclass{scrartcl}
\usepackage[xetex]{graphicx}
\usepackage{fontspec,xunicode}
\defaultfontfeatures{Mapping=tex-text,Scale=MatchLowercase}
\setmainfont[Scale=.95]{Calluna}
\setmonofont{Source Code Pro}

%\usepackage[left=3cm,right=3cm,top=1cm,nohead,driver=xetex]{geometry}

\usepackage{xltxtra}
\usepackage{hyperref}
\renewcommand{\UrlBreaks}{\do\/\do\a\do\b\do\c\do\d\do\e\do\f\do\g\do\h\do\i\do\j\do\k\do\l\do\m\do\n\do\o\do\p\do\q\do\r\do\s\do\t\do\u\do\v\do\w\do\x\do\y\do\z\do\A\do\B\do\C\do\D\do\E\do\F\do\G\do\H\do\I\do\J\do\K\do\L\do\M\do\N\do\O\do\P\do\Q\do\R\do\S\do\T\do\U\do\V\do\W\do\X\do\Y\do\Z}


%\fontspec[Path = /Users/tvooo/Library/Fonts/]{SourceCodePro-Regular.otf}

%\setmainfont[
%  Extension = .otf,
%  UprightFont = *-regular,
%  BoldFont = *-bold,
%  ItalicFont = *-italic,
%  BoldItalicFont = *-bolditalic,
%]{SourceCodePro}

\begin{document}
%\vspace{-3cm}
\title{Research Proposal}
\author{Tim von Oldenburg}


\maketitle

\section{Knowledge Area and Research Question} Since the first programmable
computers and instructions on punch cards, computing has come a long way.
Incredible progress has been made in terms of computing power, technologies and
usage areas. We have seen new programming languages and paradigms, we have seen
abstractions and whole toolchains to simplify the process of developing software.

Modern Integrated Development Environments (IDEs)\footnote{IDE: a software
integrating a code editor along with compiler, debugger, code browser and/or
other tools.} were pioneered by systems like the\\Smalltalk-80 class browser
that provided novel, domain-specific ways of navigating information (in this
case, source code). However, as Bret Victor argues in his iconic talk, not a lot
has changed since then; the progress may be described as evolutionary at best,
and includes features such as syntax highlighting, autocompletion and contextual
documentation (Victor 2013). In terms of code navigation, developers still rely
on file structures, symbol lists or keyword search.

I see large opportunities for improvement in how software is written by
improving IDEs, especially in the areas of navigating and evaluating source
code. Today's navigational patterns rely on syntax and formal data structures,
but not on semantics. Even relatively new patterns, such as presented by Deline
et al. (2006) and implemented in editors like Sublime Text\footnote{See
\url{http://www.sublimetext.com/}}, concentrate on the actual text instead of its
meaning. I want to explore how the navigation of source code can be made more
relevant and efficient if it follows \textit{semantics}, for example the
internal control flow of the executed program. While a file is a linear medium
(a set of lines, which in themselves are a set of characters), the actual program is
non-linear.

These thoughts are not completely novel, of course. But while approaches like
Visual Programming, as propagated by projects like Scratch (Maloney et al.
2010), serve a certain role, most code is still written as text. Light Table, an
experimental IDE currently in a beta testing state, explores different features
such as inline evaluation of code and spatial arrangement of semantic code
blocks (Granger 2013). I want to contribute to those experimental approaches
from the field of interaction design using proper research and validated
designs, pursuing the question of how software development can be made more
efficient through semantic and context-sensitive code editing environments.

%Integrated Development Environments (IDEs) unite editors, code browsers and
%process tools like compilers and debuggers under one hood, and are largely,
%well, integrated. They allow for editing, navigationg, and analyzing code. This
%area has seen a number of small evolutions, ranging from class browsers over
%syntax highlighting to autocompletion of code. However, programs are still
%represented as text and

%(preprocessors, debuggers, build tools, profilers)

%However, the way that software is written has largely remained the same since the invention of high-level programming languages and file systems: programs are represented as text and saved in one or multiple files.

\section{Methods} Before exploring possible design opportunities, I want to
create an overview of the status quo of relevant software development tools and
processes. A survey and personal interviews will help identify what is currently
used and popular, and will maybe even start to uncover possible shortcomings.
Besides of that, research in academic literature and contemporary development
projects will show the direction that IDEs are headed towards right now. I
expect those methods to result in an incomplete library of patterns used in
code editing environments.

The gaps in this library will be areas of opportunity to focus my further work
on. I will pick one or more of those areas (depending on the scope of work) and
design solutions for them. If possible, and feasible in a timely manner, those
designs will be created as high-level prototypes extending a real IDE (possibly
either Github’s Atom Editor\footnote{See \url{https://atom.io/}} or the Chrome DevTools\footnote{See \url{https://developers.google.com/chrome-developer-tools/}}).

These prototypes will be handed out to software developers to test. Built-in
analytics will collect metrics to serve as quantitative measures, whereas user
interviews will serve as qualitative measures of design validation. It is as
well possible to test the prototypes in a lab environment.

\section{Expected Results \& Knowledge Contribution} My aim is to provide new
knowledge of design patterns for Integrated Development Environments that lead
to more productivity and efficiency for software developers. Although I will
focus on a certain domain in software development (web development), I strive to
find insights that are applicable to a greater range of development environments
and paradigms. In addition, I want to demonstrate the practical use of said
design patterns by means of solidly implemented, high-level, usable prototypes.

%%%%%%%%%%%%%%%%%%%%%%%%%%%%
%%%%%%%%%%%%%%%%%%%%%%%%%%%%
%%%%%%%%%%%%%%%%%%%%%%%%%%%%

\section*{References}

\begin{description}
  %\item[Oney \& Brandt 2012] S. Oney and J. Brandt ``Codelets: Linking Interactive Documentation and Example Code in the Editor,'' in Proceedings of the SIGCHI Conference on Human Factors in Computing Systems, New York, NY, USA, 2012, pp. 2697–2706.
  \item[Deline et al. 2006] R. DeLine, M. Czerwinski, B. Meyers, G. Venolia, S. Drucker, and G. Robertson, ``Code Thumbnails: Using Spatial Memory to Navigate Source Code,'' in Proceedings of the Visual Languages and Human-Centric Computing, Washington, DC, USA, 2006, pp. 11–18.
  \item[Granger 2012] C. Granger ``Light Table --- a new IDE concept'',\url{http://www.chris-granger.com/2012/04/12/light-table---a-new-ide-concept/}
  \item[Victor 2013] B. Victor ``The Future of Programming'',\url{http://worrydream.com/#!/TheFutureOfProgramming}
  \item[Maloney 2010] J. Maloney, M. Resnick, N. Rusk, B. Silverman, and E. Eastmond ``The Scratch Programming Language and Environment'', Trans. Comput. Educ. 10, 4, Article 16 (November 2010)
\end{description}

\end{document}
