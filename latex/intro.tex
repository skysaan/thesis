\chapter{Introduction}\label{introduction}

Creating computer programs is a difficult and complex process.
\glspl{ide} integrate a number of tools helping software developers do
their work and are indispensable in a modern development workflow. One
class of those tools are \emph{language tools}, which help with the
actual programming language itself and reduce defects and misconceptions
\cite{hidayat}. However, research by \citeasnoun*{johnson2013} on static
analysis tools suggests that they are not as widely used, although they
proof to be helpful. This thesis attempts to approach the design of
language tools by means of interaction design methodology, through a
user-centered design process and by the example of \emph{scope}.

In programming, scope is an abstract concept to define the validity of
variables. By looking at the structure of scope, a program can be
explored from a different perspective than just its source code and
symbols, and certain pitfalls that lead to program defects can be
uncovered. In languages that implement lexical scoping, such as
JavaScript, scope analysis can be done using static analysis, and can
therefore be applied during author-time already. The research that was
done in the course of this thesis shows that scope is not yet
appropriately addressed by language tools. It is therefore used as an
example for designing, implementing and evaluating a language tool
targeting professional developers. A more in-depth explanation of scope
is given in chapter \fullref{concepts}.

\section{Process}\label{process}

This thesis project follows a \acl{ucd} process, which is as well
reflected in the structure of this document. Preliminary theoretical
groundings are presented in chapter two, which introduces software
development environments and their history and role in the development
workflow. It also presents relevant concepts of programming languages,
scope and its implications in particular. Chapter three describes the
exploration phase by means of a survey and interviews with professional
software developers to identify characteristics of well-integrated
development tools. Furthermore, canoncial and related work examples are
identified and listed, and the ideation process is exposed. The design
itself is conducted in three iterations: sketches, a scripted prototype
and a working prototype. Chapter four presents these iterations and
explains the design decisions that have been made. Integrating a solidly
implemented, high-level prototype with the Atom text editor demonstrates
the feasibility of the concept, which is further verified through user
testing. Findings from this phase are finally discussed in the fifth
chapter.

The project described in this thesis targets the needs of professional
developers with advanced experience. The final design is built for the
JavaScript programming language, but the concept of scope presents
difficulties in nearly every language in use. The knowledge gained
during the process is thus expected to be applicable to other
programming languages as well. It will be validated by means of both
quantitative and qualitative data using analytics, general feedback on
the web, and interviews.

\section{Knowledge Contribution and
Limitations}\label{knowledge-contribution-and-limitations}

This thesis explores how language tools for professional developers can
be designed and evaluated. It creates knowledge in the field of
interaction design by contributing the following:

\begin{itemize}
\itemsep1pt\parskip0pt\parsep0pt
\item
  Characteristics of well-integrated language tools that are important
  for professional developers to support their work.
\item
  A way of evaluating designs with a specific, professional target group
\item
  The implications of testing prototypes with a very limited user group
  in the open source community
\item
  Using an interaction design approach to create open source software
  opens up the field. It yields results that are most probably different
  from what typical innovation processes in open source would have
  resulted in.
\end{itemize}

In addition to the interaction design knowledge, this thesis makes the
following contributions to the open source community:

\begin{itemize}
\itemsep1pt\parskip0pt\parsep0pt
\item
  A working, extendable prototype in the form of a plug-in for the Atom
  IDE. The plug-in is released as open source.
\item
  A static analysis library to extract relevant JavaScript scope
  information. The library is written in CoffeeScript, released with the
  Atom
  plug-in\footnote{It is planned to extract the library from the plug-in in the future.}
  and can theoretically be re-used in any software to analyze scope in
  JavaScript.
\end{itemize}

\subsubsection{Limitations}\label{limitations}

The short time period in which this project was pursued (8.5 weeks)
creates some limitations. On the one hand, the final prototype—though
being a high-fidelity working prototype—has a limited set of features as
well as some bugs that influence the evaluation outcome. On the other
hand, the focus of this thesis has to be very narrow, which is why some
of the findings are difficult to apply to other programming
environments, programming languages, and less experienced developers.
