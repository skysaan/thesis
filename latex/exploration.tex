\chapter{Exploration}\label{exploration}

Following the research framework, this chapter presents the exploration
phase of the design process. Through an initial survey and a sequence of
interviews, a whole range of problems is explored. Finally, the problem
of \emph{scope} is put into focus for this thesis.

To form a general understanding of how IDEs and some of their specific
features are used, an online survey targeted towards professional
developers was created. The survey ran in April 2014 over the course of
two weeks and yielded answers from 45 participants. Besides general
questions, asking for instance which programming languages and IDEs the
participants were familiar with, it collected information about the
usage of the following IDE functionalities:

\begin{itemize}
\itemsep1pt\parskip0pt\parsep0pt
\item
  Navigation of code
\item
  Debugging
\item
  \ac{api} and language documentation
\item
  Autocompletion
\item
  Project structure and scaffolding
\item
  Asynchronicity
\item
  Syntax Highlighting
\end{itemize}

For each of the areas the survey asked if and how the participants were
using them and—if appropriate—how their IDE was supporting them. The
survey instrumented multiple-choice questions with an additional „Other“
field for custom answers, as well as open-ended questions with a
free-form text field.

\section{Survey Results}\label{survey-results}

The survey participants listed 21 different programming languages they
are using on a regular basis, as well as experience in 19 different
development environments. The participants’ backgrounds are diverse,
although the major part seems to be working with web technologies (both
front-end and back-end).

About 76\% of the participants look up \ac{api} and language
documentation mainly on the web, only a small percentage uses the
integration of documentation into IDEs. Besides that, nearly every
participant (87\%) makes use of the IDE-provided \gls{autocompletion}
feature, although most of them came up with ways to improve it. Many
comments are directed towards „smarter“, more context-aware autocomplete
suggestions, up to levels of artificial intelligence. Some comments also
mention a lack of performance and subtlety.

Navigation within large code bases is done in many different ways, such
as presented before: file browsers, symbol browsers, file search or
content search. However, there does not seem to be a clear general
preference. For structuring code, most participants rely on
platform-given modularity, for example through packages and classes in
Java. In programming languages where project structures are not given,
developers use frameworks and software design patterns to achieve a
similar structural consistency.

All participants value syntax highlighting, although for different
reasons. However, some offered suggestions on how highlighting of
certain code tokens could be used in other ways to reduce errors. Two
suggestions were targeted towards highlighting of \emph{similar}
identifiers in order to recognize typos. Others, in contrast, intended
to focus on semantics instead of syntax; for example, indicating value
changes of the \emph{this} keyword in JavaScript, highlighting the
currently focused block of code, or colour-coding the relationship of
interdependent variables.

\section{Interview Results}\label{interview-results}

In succession to the survey, ten participants agreed to be interviewed,
seven of which the author conducted interviews with. The interviewees
are either currently working as full-time or part-time professional
developers or have been doing so in the past and are now in related
positions, such as \emph{IT consultant}. Aside from that, their
backgrounds are diverse, ranging from part-time front-end developers
with a focus on design, over web application developers to low-level
audio specialists. They have experience with 12 different IDEs, using 15
programming languages on three different operating systems. Below is a
summary of interview results that are relevant to the thesis project.

Nearly all the interviewees stressed the importance of
\emph{performance} in any software development tool, especially in IDEs.
If a feature is too slow, reacts too slowly or slows the overall IDE
down, it is considered disturbing to the development workflow.
Especially the web developers praised lightweight code editors,
favouring them over the more heavyweight IDEs, but still recognizing
their drawbacks: lightweight editors are not as \emph{smart} (see
below).

Most of the interviewees also referred to their development tools of
choice in regards to the \emph{Unix philosophy}, which—according to Ken
Thompson—states that programs should „do one thing and do it well.“
\cite{raymond} This thought ultimately leads to modularily designed
systems, which was stressed in the interviews in different forms. Most
obvious is the ability to enable and disable features, as well as to
general plug-in management. Two interviewees also mentioned
\emph{modes}, although in different contexts. On the one hand, features
could run in different modes to provide help or stay out of the
developer’s way (\emph{beginner and expert modes}), on the other hand,
modes could be used to get a different perspective on a program (e.g.
highlighting of different aspects of code).

The interviewees expect their development environment to behave in a
\emph{smart} way; it should ideally know beforehand what the developer
needs in terms of support in a given situation, and what code they are
about to write. On first look, and given the current landscape of IDEs
and text editors, this contradicts the desire for a lightweight, fast,
unopinionated development environment. To be smart, a development
environment must have knowledge about the programming language, the
libraries used, and about best practices. Some IDEs are tightly
integrated with their target programming platforms, for example
Microsoft Visual Studio with the .Net platform and Eclipse with the Java
platform. But these programs are generally not considered lightweight,
performant or unopinionated. \emph{Smartness} for lightweight
environments, however, can be achieved by combining it with the modular
approach of the Unix philosophy. If specialized programming language
tools can be loosely plugged into lightweight development environments,
smartness can be achieved in those environments as well. A good example
for this is given by the numerous \emph{Linter} plug-ins for editors
like Sublime Text (see section \fullref{similar}) and projects like
CTags\footnote{See \url{http://ctags.sourceforge.net/}}.

The last relevant interview result to be mentioned in this section is a
\emph{focus on code}. No matter the target platform and the developer’s
background, code is still in focus of the development process nowadays,
which makes the code editor the most important part of any development
environment. The term \emph{inline} describes activities that happen
within the text editor itself, for example syntax highlighting. If
development tools work inline, developers do not have to switch their
focus back and forth from the authoring process. However, by putting
additional information inline, there is a risk that the text editor
becomes too cluttered or visually busy, effectively confusing and
distracting the developer. This is exemplified by pop-up windows that
block a lot of editor space or additional coloured text that makes
colour-coding ambiguous. Thus, programming tools that display
information inline must be carefully designed to be unobtrusive.

Four important characteristics for programming environments can be
extracted from the interviews: \emph{performance}, \emph{modularity},
\emph{smartness} and a \emph{focus on code}. Integrating these
characteristics is important for the usability and usefulness of
development environments, and thus for any language tools that enhance
them.

\section{Scope as a Valid Problem}\label{scope-as-a-valid-problem}

Through the conducted interviews and the survey, one can argue that
\emph{scope} is a promising and valid problem area to explore. Although
it was not referred to in the survey questions in any way, \emph{scope}
was mentioned independently by several of the survey participants and
interviewees in suggestions for the improvement of existing patterns and
tools. One of the interviewees introduced me to Crockford’s
\citeyear{crockford} approach of \emph{context colouring} (see section
\ref{similar}). A similar approach was suggested in the survey in the
context of editing. Though not necessarily related to scope, the
participant suggested to highlight the current code block the cursor is
placed in; this is already done by some editors and IDEs, and is adapted
in the concept presented in this thesis as well. Another interviewee
suggested to indicate changes of the
\texttt{this}\footnote{In JavaScript, the keyword \texttt{\gls{this}} refers to the current execution context.}
context in JavaScript, which is closely related to scope, although being
run-time specific.

The strongest alternative problem to possibly focus on was
\emph{asynchronicity} and the writing and debugging of asynchronous
code. After some research on the topic, I found the work of
\citeasnoun{lieber} to be quite substantial and possibly parallel to my
then-prospective work. Lieber implements \emph{Theseus}, an asynchronous
JavaScript debugger, and will discuss it from an Interaction Design
perspective in his forthcoming master’s thesis. This is why I did not
choose the topic of asynchronicity, but instead focused on the problem
of scope.
