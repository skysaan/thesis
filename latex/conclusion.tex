\chapter{Reflection}\label{reflection}

The following chapter discusses the results of the user testing
(evaluation) with respect to the goals presented in the introduction. It
also reflects on the design process and makes suggestions for
improvements in both the process and the concept.

\section{Quantitative Reflection}\label{quantitative-reflection}

From a quantitative standpoint, it is hard to tell if the \emph{Scope
Inspector} package is useful or successful. The following two metrics
are taken into consideration.

\begin{description}
\item[Number of downloads]
The number of downloads, as presented by the package
website\footnote{See \url{https://atom.io/packages/scope-inspector}}, is
a metric that has to be compared to others in order to drive a
meaningful conclusion. The total number of Atom installations is unknown
to me, however, it is possible to make an estimate: The package
\emph{Find and
Replace}\footnote{https://atom.io/packages/find-and-replace} is part of
Atom’s standard distribution and thus should have been downloaded about
the same amount of
times\footnote{This estimate does not include custom builds of Atom, for example by compilation of the source on Windows or Linux.}.
By the time of writing, \emph{Find and Replace} has been downloaded
14.000 times. Compared to this, the number of downloads for \emph{Scope
Inspector} seems low, although one has to take different circumstances
into consideration:

\begin{enumerate}
\def\labelenumi{\arabic{enumi}.}
\itemsep1pt\parskip0pt\parsep0pt
\item
  The Atom editor, and thus the \emph{Find and Replace} package, has
  been around much longer than the \emph{Scope Inspector} package. The
  first release is dated to September 17,
  2013\footnote{See \url{https://github.com/atom/find-and-replace/tree/v0.2.0}}.
  Thus, the package has been around for about 8 months, while this
  prototype has only been around for two weeks by the time of writing.
\item
  The \emph{Find and Replace} package provides a core functionality of
  every text editor, whereas the \emph{Scope Inspector} package fulfills
  a specialized task for a narrow target group.
\item
  The \emph{Find and Replace} package comes with each Atom installation.
  However, to even \emph{get to know} about the Scope Inspector package,
  one has to search for or stumble upon it.
\end{enumerate}

Based on these arguments, one should find a package with more similar
characteristics to do a comparison. The one that comes closest in terms
of target group and functionality is the JSHint plug-in for the Linter
package, which provides a JavaScript linting
tool\footnote{See \url{https://github.com/AtomLinter/linter-jshint/tree/v0.0.1}}.
From its first release on April 18, 2014 until the time of writing, the
package has been installed about 4900 times, which is about 30 times the
number of downloads of Scope Inspector in twice the time. Given the
familiarity of linting tools, which are part of most JavaScript
developers’ workflows, and the novelty of Scope Inspector, one can
conclude that the Scope Inspector’s number of downloads is moderately
high. However, while the number of downloads may proof that there exists
some interest in this sort of language tool, it is not a sign for its
usefulness.
\item[Analytics]
Including me, analytics have only been enabled by five users. There may
be a couple of reasons for this low number.

\begin{enumerate}
\def\labelenumi{\arabic{enumi}.}
\itemsep1pt\parskip0pt\parsep0pt
\item
  Analytics tracking is disabled by default for ethical reasons: I do
  not want to track users without them knowing and explicitly giving
  their consent.
\item
  Users do not want to opt-in to analytics tracking, or consider it to
  be unimportant. In this case, a stronger point for the benefits of
  tracking for both the user and me should be made.
\item
  The package is only used by a very low number of users. This is
  likely, but the difference between the five users who enabled
  analytics and the 150 who downloaded it seems to be too high—the real
  number of users is probably in the middle of 5 and 150.
\item
  The request to opt-in is too subtle and not visible enough. It is
  quite possible that users do not read the README file
  thorougly\footnote{The README file is commonly used in open source projects to communicate a project’s purpose and even document it. Against many presumptions, it is also usually read.}.
  In this case, the option to opt-in has to be made more prominent. It
  can be moved up in the README file and emphasized better. Another
  approach would be to advertise the tracking (along with this research
  project) in the package-itself, for example by showing an information
  window on first activation of the package.
\item
  The tracking code is broken. This is not very possible, as in this
  case, the analytics tool would not have collected any data.
\end{enumerate}

It is likely that opt-in by default would have raised the number of
tracked users significantly. However, as I stated before, this would be
strongly against my ethical position and I am convinced that there are
other ways to gain relevant testing data. The second most promising
approach is to raise the visibility of the opt-in option and market the
research better, as stated above in (2) and (4).

With the low number of analytics opt-ins, it is hard to draw a
conclusion on the usefulness of the package and its single components.
However, as the sidebar was being enabled each day of the testing period
with the exception of one, it can be claimed that the package was
actively used by all of the opted-in users.
\end{description}

\section{Qualitative Reflection}\label{qualitative-reflection}

More meaningful than the quantitative metrics are qualitative results.
The original goal was to evaluate the prototype by distributing it to
the users and let three developers use it for at least one week (or one
day, respectively) in a real working environment. However, as mentioned
in the previous chapter, the recruited developers were unable to use the
prototype in their daily work, because they were mainly developing with
\ac{html} and \ac{css} during that time period. For that reason, I had
to rely on other ways of testing: remote interviews and unstructured
feedback via social media channels.

\begin{description}
\item[Remote interviews]
The testing scenario was not optimal due to the following circumstances:

\begin{enumerate}
\def\labelenumi{\arabic{enumi}.}
\itemsep1pt\parskip0pt\parsep0pt
\item
  All users except one were only available for remote testing. The one
  available locally was using JavaScript only casually and not on a
  daily basis. It would have certainly been more fruitful to work with
  professional developers locally in Malmö. However, I failed to recruit
  developers in the short period of time available—especially because of
  the narrow target group of the concept—and thus relied mostly on
  developers I knew from previous work in Germany. With a less
  restricting time limit, this circumstance could most possibly be
  improved upon.
\item
  None of the users who was available for an interview was able to test
  the prototype during a whole week in a real working environment.
  Instead, only in situ testing with prepared and existing source code
  could be conducted. Given more time, it might have been possible to
  recruit more full-time developers. Also, the three developers already
  recruited could have participated in testing at a later point in time.
\end{enumerate}

Despite the unfortunate circumstances, the feedback gathered through
remote interviews is detailed and useful. It can be used as a solid
grounding to enhance the prototype in future iterations. The only point
where the feedback comes short is in long-term testing—the question how
well the prototype can be integrated into existing development workflows
cannot be answered satisfactorily.
\item[Social Media Channels]
Social Media have been used marketing as well as feedback channels.
Announcing the Scope Inspector on Twitter, EchoJS and Reddit contributed
significantly to its distribution. The feedback yielded on Twitter and
Github—exclusively from users who are unknown to me—shows that social
media are suitable channels to communicate about and announce open
source projects. For future projects it will be a good idea to ask
influencers on those media—JavaScript experts with a lot of
connections—to look at and write about it. This way, even more traction
could be gained.
\end{description}

Atom—along with \ac{apm}—is a new platform with a growing community,
which is still in its early phase. The impacts of these circumstances
have been described in detail in this chapter. Whereas I expected the
social media coverage to be lower than it was, the actual user testing
did not go without problems.

For similar projects in the future, it thus seems best to choose a more
mature platform—one with more users, available on more operating
systems, and with an extensive ecosystem of plug-ins, knowledge, and
community. A good example for a project on such a platform is Theseus by
\citeasnoun{lieber}. Based on Brackets, which is around way longer than
Atom and more widely distributed (comparing the activity on Github), it
had a better chance and more time to be adapted. On the other hand, the
momentum of growing platforms like Atom—especially if it is backed by
one of the biggest open source communities (Github)—should not be
underestimated. The Scope Inspector will most likely benefit from this
in the future.

\chapter{Conclusion \& Outlook}\label{conclusion-outlook}

While the disciplines of Interaction Design and Software Development are
deeply intertangled, the world of IDEs and programming language tools is
still mainly driven by technology. This thesis made an approach to apply
interaction design methodology to this area by designing a language tool
for a very specific, narrow target group.

\section{Knowledge Contribution}\label{knowledge-contribution}

Through the design process, this thesis contributes knowledge to the
interaction design community. Four characteristics of well-integrated
programming language tools—performance, modularity, smartness, and focus
on code—have been identified. Those characteristics can be used as a
loose grounding to evaluate language tools for professional software
developers. Other target groups may prioritize different
characteristics.

The way that the user testing was planned and prepared seemed, at that
time, reasonable. Due to my background in web development, I recruited
professional developers I knew for user testing. However, the fact they
could not make use of the prototype due to work reasons was
unforeseeable for me. The subsequent attempt to recruit users through
the survey failed as well—most users knew JavaScript to some extent, but
for none was it the primary professional language. Given the fact that
users would need to work with Atom, which only runs on the OS X platform
by the time of writing, problems with user testing could have been
expected. Apparently, testing prototypes with a very limited user group
in the open source community implies that great efforts have to be taken
to recruit users, and longer time periods are necessary than have been
given in this case. But given enough time and access to more test users,
this approach of evaluating designs in the field—with analytics,
long-term tests, and social media—is promising.

Picking \emph{scope} as a topic of focus, I was quite surprised not to
find any integrated language tool that addresses scope in the way my
design does—especially as the survey and interviews identified it easily
as an ongoing issue in programming. This leads to the assumption that
user-centered design methods, as were applied in this project, do not
usually lead innovation in open source projects like there. A quick look
into several open source
projects\footnote{User-driven innovation processes in open source software are subject to a whole other line of research, and are thus not discussed in detail here.}
of different size shows that user participation exists in such projects,
but is realized in a much looser way, for example through whole
ecosystems of institutions and communities, as in the case of Linux
\cite{raymond2}; through suggestion platforms, as in the case of
Mozilla\footnote{See for example: \url{https://hacks.mozilla.org/2014/05/developer-tools-feedback-channels-one-week-in/
}}; or through open contribution, as in the case of the masses of
smaller open source projects and made possible through platforms like
Github and Launchpad\footnote{See \url{https://launchpad.net/}}.
\citeasnoun{raymond2} states that every good work of software starts by
scratching a developer's personal itch. However, the user-centered
design process forces the designer to look for \emph{other developers’
itches}. Thus I think that the results of this thesis differ from what a
classical open source approach may have resulted in. Applying
interaction design methodologies proved to be highly beneficial, and
will probably be so for other open source projects. While this knowledge
contribution was not expected, it is very welcome.

\section{Applicability}\label{applicability}

Scope is a concept that appears in most programming languages. While
JavaScript was taken as an example throughout the process, the design is
transferable to other languages as well. Closest to JavaScript are, of
course, other languages that implement functional paradigms and similar
scoping rules, for example Clojure, Dart, or Python. Nevertheless, today
even originally non-functional languages are starting to implement
features like closures, most recently Java and PHP. But even without
functional features, scope remains a problem for developers of many
programming languages, beginners and professionals alike. The concept
created in the course of this thesis can be beneficial to them, as well.

Additionally, the concept is quite independent from the development
environment. Though it was using the visual language and interaction
models of Atom, those can be as easily adapted to IDEs like Eclipse,
Visual Studio, or IntelliJ, as they can be to Sublime Text, Vim, or
Emacs.

\section{Outlook}\label{outlook}

The feedback gathered through user testing is most helpful in driving
the Scope Inspector forward. I look forward to see how it will be
adapted by the community of JavaScript developers as it starts to
implement more relevant features, such as closure detection and sidebar
navigation. More time will provide the long-term testing that this
thesis project could not, and the prototype’s value will be proven or
disproven in the long run. In the meantime, I am excited to see if and
how interaction design methodology further drives innovation in the open
source community.
